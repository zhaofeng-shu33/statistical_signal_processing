\documentclass{article}
\usepackage{ctex}
\ctexset{
  proofname = \heiti{证明}
}
\usepackage{amsmath,esint,amssymb}
\usepackage{hyperref}
\usepackage{graphicx}
\usepackage{bm}
\usepackage{mathtools}
\usepackage{esdiff}
\DeclareMathOperator*{\rgmax}{argmax}
\DeclarePairedDelimiter\abs{\lvert}{\rvert}
\usepackage[thehwcnt = 2]{iidef}
\thecourseinstitute{清华大学深圳研究生院}
\thecoursename{统计信号处理}
\theterm{2017年春季学期}
\hwname{作业}
\slname{\heiti{解}}
\DeclareMathOperator\sign{\mathrm{sgn}}
\begin{document}
\courseheader
\name{赵丰}
\begin{enumerate}
\item 设一次观测值为 $ z = ab +n $,其中 $ a, b $ 和 $n $ 是均值为零, 方差分别为 $ \sigma_a^2, \sigma_b^2 $ 
和  $ \sigma_n^2 $ 且相互统计独立的高斯随机变量。 若对 $ a $ 和 $ b $ 同时进行估值, 试求 $ a $ 和 $ b $ 
的最大后验估值。
\begin{solution}
$p(a, b | z) \propto p(a)p(b) p(z | a,b) $。 为极大化 $ p(a,b | z)$,只需极小化
$ L(a, b)= { ( z - ab)^2 \over \sigma_n^2 } + { a^2 \over \sigma_a^2} + { b^2 \over \sigma_b^2} $
分别对 $a, b $ 求偏导数得:
\begin{align*}
{1 \over 2}\diffp{L(a, b)}{a}  & = { a \over \sigma_a^2 } - { b(z-ab) \over \sigma_n^2 } = 0 \\
{1 \over 2}\diffp{L(a, b)}{b} & = { b \over \sigma_b^2} - {a(z-ab) \over \sigma_n^2} = 0
\end{align*}
由 $ab$乘积的对称性可知无法区分$a$和 $b$的符号,因此我们不妨设 $ a>0$。
两等式联立消元得 $ { a \over \sigma_a } = \pm{ b \over \sigma_b }$。 若 $ b>0$,解得$a=b=0$或
\begin{equation}\label{eq:case1}
{ a^* \over \sigma_a } = { b^* \over \sigma_b } = {\sqrt{\sigma_a \sigma_b z - \sigma_n^2} \over \sigma_a \sigma_b}
\end{equation}
若 $ b<0 $, 解得$a=b=0$或
\begin{equation}\label{eq:case2}
{ a^* \over \sigma_a } = { -b^* \over \sigma_b } = {\sqrt{-\sigma_a \sigma_b z - \sigma_n^2} \over \sigma_a \sigma_b}
\end{equation}
并且$L(a,b)$ 的Hessian 矩阵满足 
$$
{1 \over 2} H(a,b) = \begin{bmatrix} {1 \over \sigma_a^2 } + { b^2 \over \sigma_n^2 } & {2ab-z \over \sigma_n^2} 
\\ {2ab-z \over \sigma_n^2} & {1 \over \sigma_b^2} + {a^2 \over \sigma_n^2} \end{bmatrix}
$$
在$(0,0)$处正定。
代入~\eqref{eq:case1} 或 ~\eqref{eq:case2} 式,可求出Hessian 矩阵的行列式为:
\begin{equation}
H(a^*,b^*) = {16 \over \sigma^2_a\sigma^2_b\sigma^2_n}(\sigma_a\sigma_b\abs{z} - \sigma_n^2)>0
\end{equation}
因此$(0, 0)$ 和 $(a^*, b^*)$ 均为极小值点。
另一方面,$L(0, 0) = { z^2 \over \sigma_n^2}$,
代入~\eqref{eq:case1} 或~\eqref{eq:case2}式得 $L(a^*, b^*) = {2 \abs{z} \over \sigma_a \sigma_b} - {\sigma_n^2 \over \sigma_a^2 \sigma_b^2}$。~\eqref{eq:case1} 或~\eqref{eq:case2}式成立的前提条件是 $ {\abs{z} \over \sigma_n } > {\sigma_n \over \sigma_a \sigma_b} $。 由$ ({\abs{z} \over \sigma_n } - {\sigma_n \over \sigma_a \sigma_b} )^2> 0$ 可得 $L(0, 0) > L(a^*,b^*) $。

因此当 $\abs{z} \leq {\sigma_n^2 \over \sigma_a \sigma_b } $时, $ a $ 和 $ b $ 的最大后验估值均为0;当
$\abs{z} > {\sigma_n^2 \over \sigma_a \sigma_b } $时,$a$和 $b$的最大后验估值为:
$$
{ a^* \over \sigma_a } = { \sign(z b^*) \over \sigma_b } = {\sqrt{\sigma_a \sigma_b \abs{z} - \sigma_n^2} \over \sigma_a \sigma_b}
$$


\end{solution}
\item 设一物体作自由下落,在 $ t $ 秒内下降距离 $ s(t) = { 1 \over 2} g t^2 (m)$。现有一台有噪声的仪器进行观测,
以估计重力加速度 $ g (m/s^2) $。其观测模型为:
$$
 z_i = \frac{ i^2}{2} g + n_i , i = 1, 2, \dots
$$
已知 $ \E[g] = g_0 (m/s^2), \Var[g] = 1 (m^2/s^4), \E[n_i] =0, \E[n_i n_j] = \left({1 \over 2}\right)^{\abs{j-i}}, \E[g n_i]=0 $
\begin{enumerate}
\item 取一次采样 $ z_1 = { 1 \over 2} g + n_1 $, 求重力加速度 $ g $ 的线性最小均方估计。
\item 取两次采样 $ z_1 = { 1 \over 2} g + n_1, z_2 = 2g + n_2 $,求重力加速度 $ g $ 的线性最小均方估计。
\item 比较(a) 和 (b) 两次估计的质量。
\end{enumerate}
\begin{solution}
\begin{enumerate}
\item $ \E[z_1] = { 1 \over 2} g_0, \E[ g ] = g_0, \Cov[z_1, g] = { 1 \over 2}, \Var[z_1] = { 5 \over 4 }  $
根据公式:$$ \hat{\bm{\theta}}_{\textrm{LMS}}=\Cov[\bm{\theta},\bm{z}]\Var^{-1}[\bm{z}](\bm{z}-\E[\bm{z}])+\E[\bm{\theta}] $$
可得 $ \hat{g} (z) = { 2 \over 5 } ( z - {1 \over 2} g_0 ) + g_0 $
\item 设 $ \bm{z} = [z_1, z_2]^T, \Cov[g,\bm{z}] = [{ 1 \over 2}, 2], \Var[\bm{z}] = \begin{bmatrix} { 5 \over 4} & { 3 \over 2} \\ {3 \over 2} & 5
\end{bmatrix} \Rightarrow \hat{g} = -{ 1 \over 8} ( z_1 - { 1 \over 2} g ) + {7 \over 16} ( z_2 - 2 g) + g_0$
\item 根据公式 $\Var[\bm{\theta}_{\textrm{LMS}}-\bm{\theta}]  = \Var[\bm{\theta}] - \Cov[\bm{\theta},\bm{z}]\Var^{-1}[\bm{z}]\Cov[\bm{z},\bm{\theta}]$,分别计算可得 (a) 中估计的均方误差为 $ { 4 \over 5 } $,(b) 中估计的均方误差为 $ { 3 \over 16 } $。因此 (b) 的平均误差比
(a) 的小。
\end{enumerate}
\end{solution}
\item 设有一阶动态系统模型。状态方程为 $ \dot x(t) = -x(t) + w(t) $ ,观测方程为  $ z(t) = x(t) + v(t) $。 其中 $ w(t) $  和 $ v(t) $ 是零均值、互不相关的白噪声过程。
假设 $ \Cov[w(t), w(\tau)] = 2 \alpha \delta(t - \tau), \Cov[v(t), v(\tau) ] = \alpha \delta(t - \tau) $,这里 $ \alpha $ 为常数。
试求:
\begin{enumerate}
\item 达到稳态时的 Kalman 滤波器。
\item 信号 $x(t) $ 的物理可实现的Wiener 滤波器。
\item 比较和分析前面所得的结果。
\end{enumerate}
\begin{solution}\mbox{}
\begin{enumerate}
\item 由于求稳态解,令 $ \dot P(t) = 0 $ 。关于误差方差的 Ricatti 方程为 $ P^2 + 2 \alpha P = 2 \alpha^2 $。舍去负根有 
$ P = (\sqrt{3} - 1) \alpha$。 从而滤波增益为 $ K = \sqrt{3} - 1 $。进而得到状态估计方程为:
$\diff{\hat{x}(t)}{t} + \sqrt{3} \hat{x}(t) = (\sqrt{3} - 1) z(t)$,滤波器传递函数为  $H(s) = { \sqrt{3} - 1 \over \sqrt{3} + s } $。
\item 由系统的状态方程可得形成滤波器为 $ H(s) = { 1 \over 1 + s} $。信号 $x(t)$ 的 PSD 为 $\Phi_x(s) =  H(s) H(-s) \Phi_w(s) = { 2\alpha \over 1 - s^2 }$。噪声 $ \Phi_v(s) = \alpha $。根据物理可实现的 Wiener 滤波器的公式,
$ H(s) = \frac{1}{\Phi_z^+(s)}\left[\frac{\Phi_{xz}(s)}{\Phi_z^{-}(s)}\right]^{t+}$ 可求出 $ H(s) ={  \sqrt{3} - 1 \over \sqrt{3} + s }$, 进而求出滤波器的冲击响应为  : $ h(t) = (\sqrt{3} - 1) e^{-\sqrt{3} t} u(t) $。
\item 稳态时的 Kalman 滤波器与物理可实现的Wiener 滤波器形式相同。
\end{enumerate}
\end{solution}
\item 试证明:
对于 Wilcoxon 检验: $ \diffp{\E[T^+]}{\theta} \big\vert_{\theta = 0} = n f_0(0) + n(n-1) I $。
其中 $ I = \int_{-\infty}^{+\infty} f_0^2(\sigma) d\sigma $
\begin{proof}
设 $ f(\sigma)$ 关于$y$轴对称。
可以求出 $ T^+  = \sum_{ i = 1 }^n \sum_{ j = 1}^i u(X_i + X_j) $

当 $ i \neq j $时,  $ \E[u(X_i + X_j) ] = 1 - \int_{ -\infty }^{ +\infty} f_0(\sigma - \theta) F_0( - \sigma - \theta ) d\sigma $。
进一步,
\begin{align*}
\diffp{\E[u(X_i + X_j) ] }{\theta} & = \int_{ -\infty }^{ +\infty} \diffp{f_0(\sigma - \theta)}{\sigma} F_0( - \sigma - \theta ) d\sigma 
+ \int_{ -\infty }^{ +\infty} f_0(\sigma - \theta) f_0( - \sigma - \theta ) d\sigma \\
& = 2\int_{ -\infty }^{ +\infty} f_0(\sigma - \theta) f_0( - \sigma - \theta ) d\sigma  
\end{align*}
从而有 $ \diffp{\E[u(X_i + X_j) ] }{\theta}\big\vert_{\theta = 0} = 2I $。

当 $ i = j $ 时, $ \E[u(X_i + X_j)] = 1 - F_0(-\theta) \Rightarrow  \diffp{\E[u(X_i + X_j) ] }{\theta}\big\vert_{\theta = 0} = f_0(0) $。
因此,$ \diffp{\E[T^+]}{\theta} \big\vert_{\theta = 0} = n f_0(0) + n(n-1) I $。
\end{proof}
\end{enumerate}

\end{document}
