\documentclass{ctexart}
\ctexset{
proofname = \heiti{证明}
}
\usepackage{amsmath,amsthm,amsfonts,amssymb}
\usepackage{enumitem}
\usepackage{esdiff}
\usepackage{pifont}

\newtheorem{definition}{定义}
\newtheorem{thm}{定理}
\newtheorem{pro}{命题}
\newtheorem{cor}{推论}
\newtheorem{example}{例}
\newtheorem*{solution}{解}
\def\P{\textbf{P}}
\def\E{\mathbb{E}}
\def\R{\mathbb{R}}
\def\Z{\mathcal{Z}}
\def\Var{\textrm{Var}}
\def\Cov{\textrm{Cov}}
\usepackage{mathtools}
\DeclareMathOperator*\argmin{\arg\!\min}
\DeclarePairedDelimiter\abs{\lvert}{\rvert}
\usepackage{bm}
\begin{document}
\title{连续时间 Kalman 滤波}
\author{赵丰}
\maketitle
连续时变系统模型:

连续的状态方程---
\begin{equation}\label{eq:state}
\dot X(t) = A(t) X(t) + B(t) U(t)
\end{equation}

连续的观测方程---
\begin{equation}\label{eq:observation}
Z(t) = H(t) X(t) + V(t)
\end{equation}

$ \E[U(t)] = 0, \E[U(t) U^T(\tau)] = Q(t) \delta(t-\tau)$

$ \E[V(t)] = 0, \E[V(t) V^T(\tau)] = R(t) \delta_{t-\tau}, \E[U(t) V^T(\tau)] = 0$

已知初始状态 $X(t_0)$ 均值为 $\E[X(t_0)] = \mu_X(t_0)$, 方差为 $\Var(X(t_0)) = P(t_0)$, 与 $U(t), V(t)$ 均不相关。

由~\eqref{eq:state} 式解得
$$
X(t) = \Phi(t, t_0) X(t_0) + \int_{t_0}^t \Phi(t, \tau) B(\tau) U(\tau) d\tau
$$
其中 $\Phi(t, t_0) = \exp( \int_{t_0}^{t} A(\tau) d\tau)$.

先将连续系统化为离散系统, $ t = k \Delta t, t_0 = (k-1)\Delta t, 
X_k = X(k\Delta t), X_{k-1} = X((k-1)\Delta t), \Phi_{k, k-1} = \Phi(k \Delta t, (k-1) \Delta t) $,
则有
$$
X_k = \Phi_{k, k-1} X_{k-1} + 
W_{k-1}
$$
其中 
\begin{align*}
W_{k-1} & = \int_{(k-1) \Delta t}^{k \Delta t} \Phi(k\Delta t, \tau) B(\tau) U(\tau) d\tau \\
& \approx \Phi(k\Delta t, (k-1)\Delta t) B((k-1)\Delta t) \int_{(k-1)\Delta t}^{k \Delta t} U(\tau) d\tau
\end{align*}
$\E[W_k] = 0,$
并可以求出 
$ \E[W_k W_i^T] = Q_k \delta_{ki}, $
\begin{equation}\label{eq:Q_k}
Q_k = \Phi(t+\Delta t, t) B(t) Q(t) B^T(t) \Phi^T(t+\Delta t, t)\Delta t
\end{equation}

对~\eqref{eq:observation} 式两边在$(t, t+\Delta t)$区间积分再除以 $\Delta t$, 令 $Z_k = z(t), H(k) = H(t), X_k = X(t)$,
则 
$$
Z_k = H_k X_k + V_k
$$
其中 $V_k = {1 \over \Delta t} \int_{k\Delta t}^{(k+1) \Delta t} V(\tau) d\tau $, 且 $\E[V_k] = 0 $
并可以求出
$ \E[V_k V_i^T] = R_k \delta_{ki}, R_k = { R(t) \over \Delta t} $

写出相应的离散 Kalman 滤波递推公式为
\begin{equation}\label{eq:discrete_update}
\hat{X}( t + \Delta t) = \Phi(t+\Delta t, t) \hat{X}(t) + K(t+\Delta t) [ Z(t+\Delta t) - H(t+ \Delta t) \Phi(t+\Delta t, t) \hat{X}(t)]
\end{equation}
因为 $\Phi(t+\Delta t, t) = \exp(A \Delta t) = 1 + A(t) \Delta t + O( (\Delta t)^2)$
\begin{align*}
\diff{\hat{X}(t)}{t} & = \lim_{\Delta t \to 0} \frac{ \hat{X}(t + \Delta t) - \hat{X}(t) }{\Delta t} \\
& = \lim_{\Delta t \to 0} \frac{1}{\Delta t} [\Phi(t + \Delta t, t) - 1] \hat{X}(t)  \\
& + \lim_{\Delta t \to 0} \frac{1}{\Delta t} K(t+\Delta t) [ Z(t+\Delta t) - H(t + \Delta t) \Phi(t+\Delta t, t) \hat{X}(t)] \\
& = A(t)\hat{X}(t) + \lim_{\Delta t \to 0} \frac{ K(t + \Delta t)}{\Delta t} [ Z(t) - H(t) \hat{X}(t) ]
\end{align*}
由相应的离散 Kalman 滤波增益矩阵公式:
\begin{equation}\label{eq:discrete_gain_matrix}
K( t + \Delta t) = P(t+\Delta t) H^T(t+\Delta t) R^{-1}( t + \Delta t) \Delta t
\end{equation}
设 $ K(t) = \lim_{\Delta t \to 0} \frac{K(t+\Delta t)}{\Delta t} $, 则我们得到连续情形下的滤波增益方程:
\begin{equation}\label{eq:continuous_gain_matrix}
K(t)  = P(t) H^T(t) R^{-1}(t)
\end{equation}
以及连续情形下的状态估计方程:
\begin{equation}\label{eq:continous_state}
\diff{\hat{X}(t)}{t} = A(t) \hat{X}(t) + K(t) [ Z(t) - H(t) \hat{X}(t) ] 
\end{equation}
由相应的离散 Kalman 滤波误差矩阵的递推公式:
\begin{align*}
P(t+\Delta t) & = [I - K(t+\Delta t) H(t+ \Delta t) ] (\Phi(t+\Delta t, t) P(t) \Phi^T(t + \Delta t, t) + Q_k) \\
& = [I - K(t+\Delta t) H(t+ \Delta t) ] (P(t) + A(t)P(t)\Delta t + P(t)A^T(t) \Delta t + Q_k +O( (\Delta t)^2)) \\
\end{align*}
因为 $K(t+\Delta t) = O(\Delta t)$
我们进一步有
$$
P(t+\Delta t) =  P(t) + A(t)P(t)\Delta t + P(t)A^T(t) \Delta t + Q_k - K(t+\Delta t) H(t + \Delta t) P(t) + O( (\Delta t)^2)) 
$$
由~\eqref{eq:Q_k}式可得
$ \lim_{\Delta t \to 0} \frac{Q_k}{\Delta t} = B(t)Q(t)B^T(t) $
因此,结合~\eqref{eq:discrete_gain_matrix}式
\begin{align}
\diff{P(t)}{t} & = \lim_{\Delta t \to 0} \frac{P(t + \Delta t) - P(t)} {\Delta t} \\
\label{eq:Ricatti} & = A(t)P(t) +P(t)A^T(t) + B(t)Q(t)B^T(t) -  P(t)H(t)^T R^{-1}(t) H(t)P(t)
\end{align}
~\eqref{eq:Ricatti} 式为误差方差方程,也称为 Ricatti 方程。

我们用连续时间 Kalman 滤波针对加性噪声信道设计滤波器。
设 $ z(t) = s(t) + n(t)$。 信号的 PSD 为 $ \Phi_s(s) = { 2 \over 1-s^2}$, 设噪声为高斯白噪声, $\Phi_n(s) = 1 $。 则我们可以得到形成滤波器为
$ H(s) = { \sqrt{2} \over s+1}$ 因此状态 方程可写为 
$$ 
\dot s(t) = -s(t) + \sqrt{2} u(t) 
$$
观测方程为仍为$ z(t) = s(t) + n(t)$
由 ~\eqref{eq:continuous_gain_matrix} 可得到 $ K(t) = P(t)$。我们寻找稳态解,令~\eqref{eq:Ricatti}中$\dot P(t) = 0$, 舍去负根得 $ P = \sqrt{3} - 1 $。 所以
由~\eqref{eq:continous_state}可得:
$$
\diff{\hat{x}(t)}{t} + \sqrt{3} \hat{x}(t) = (\sqrt{3} - 1) z(t)
$$
滤波器传递函数为  $H(s) = { \sqrt{3} - 1 \over \sqrt{3} + s } $, 与物理可实现的 Wiener 滤波器相同。
\end{document}



