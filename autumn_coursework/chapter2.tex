\documentclass{article}
\usepackage{ctex}
\ctexset{
  proofname = \heiti{证明}
}
\usepackage{amsmath,esint,amssymb}
\usepackage{hyperref}
\usepackage{bm}
\DeclareMathOperator*{\rgmax}{argmax}
\usepackage[thehwcnt = 2]{iidef}
\thecourseinstitute{清华大学}
\thecoursename{现代信号处理}
\theterm{2017年秋季学期}
\hwname{作业}
\slname{\heiti{解}}
\begin{document}
\courseheader
\name{赵丰}
\begin{enumerate}
\item 最常用的随机信号均值估计器是 :$\hat{\mu}_x = \frac{1}{N} \sum_{n=0}^{N-1} x(n)$。证明,如果$x(n)$ 不是白噪声,估计器的方差为

\begin{equation}\label{eq:to_proof}
\Var(\hat{\mu}_x)=\frac{1}{N}\sum_{l=-N}^N (1-\frac{|l|}{N})c_x(l)
\end{equation}

这里 $c_x(l)$ 是 $x(n)$ 的协方差函数。
\begin{proof}
设$y(i)=x(i)-\E[x(i)]$,则$y(i)$是零均值的
随机变量, 且有$\Var(\hat{\mu}_x)=\Var(\hat{\mu}_y),c_x(l)=c_y(l)$,因此我们只需证明:
\begin{equation}
\Var(\hat{\mu}_y)=\frac{1}{N}\sum_{l=-N}^N (1-\frac{|l|}{N})c_y(l)
\end{equation}
由定义
\begin{align}
\Var(\hat{\mu}_y) =& E\{( \frac{1}{N} \sum_{n=0}^{N-1} y(n) )^2\} \\
 =& \frac{1}{N^2} \sum_{m=0}^{N-1}\sum_{n=0}^{N-1} E(y(n)y(m)) \\
 =& \frac{1}{N^2} \sum_{m=0}^{N-1}\sum_{n=0}^{N-1} c_y(n-m), \text{设$l=n-m$} \\
 =& \frac{1}{N^2} \sum_{m=0}^{N-1}\sum_{l=-m}^{N-1-m} c_y(l) \label{eq:1A}
\end{align}
对于给定的$l$,讨论$m$所有可能取值的个数为满足如下两个不等式约束的所有整数:
\begin{equation}
\begin{cases}
-l\leq m \leq &  N-1-l \\
0 \leq m \leq & N-1 \\
\end{cases}
\end{equation}
分$l$是否大于零讨论:若$l\geq 0$,则有$0\leq m \leq N-1-l$,共有$N-l$中取值;
若$l<0$,则有$-l\leq m \leq N-1$,共有$N+l$种取值,综合来看对固定的$l$,$m$共有$N-|l|$种取值,
又因为$-(N-1)\leq l \leq N-1$,所以我们把 式\eqref{eq:1A}交换求和次序先对$l$求和再对$m$求和
可以得到
\begin{equation}
\Var(\hat{\mu}_y)= \frac{1}{N^2} \sum_{l=-(N-1)}^{N-1} (N-|l|)c_y(l)
\end{equation}
上式整理后即得式\eqref{eq:to_proof}。
\end{proof}
\item
通过观察序列 $x(n) = s(n,\theta) + w(n), n= 0,1,\dots,N-1,w(n)\sim N(0,\sigma^2)$是高斯白噪声,$s(n,\theta)$ 是 $\theta$ 的函数并可导,证明 $\theta$ 的无偏估计器$\hat{\theta}$满足:
\begin{equation}\label{eq:CRbound}
\Var[\hat{\theta}] \geq \frac{\sigma^2}{\sum_{n=0}^{N-1} \left(\frac{\partial s(n,\theta)}{\partial \theta}\right)^2}
\end{equation}
\begin{proof}
利用克拉美罗界的性质,只需求解参数$\theta$的Fisher 信息。
由Fisher 信息的公式
\begin{equation}
I(\theta)=-E(\frac{\partial^2 \log(p(\bm{x},\theta))}{\partial \theta^2})
\end{equation}
首先求随机变量$X$的概率密度函数, 是均值为$s(n,\theta)$,方差为$\sigma^2$的高斯分布的概率密度函数的乘积,因此其
对数似然函数为:
\begin{equation}
\log(p(\bm{x},\theta))=-\frac{N}{2}\log(2\pi \sigma^2)-\sum_{n=0}^{N-1} \frac{(x_n-s(n,\theta))^2}{2\sigma^2}
\end{equation}
因此
\begin{align}
I(\theta)= &\frac{1}{\sigma^2}\sum_{n=0}^{N-1}\E\left[ \biggl(s(n,\theta)-x_n\biggr)\frac{\partial^2 s(n,\theta)}{\partial \theta^2}+\left(\frac{\partial s(n,\theta)}{\partial \theta}\right)^2\right]\\
=& \sum_{n=0}^{N-1}\frac{1}{\sigma^2}\left(\frac{\partial s(n,\theta)}{\partial \theta}\right)^2
\end{align}
最后我们得到
\begin{align}
\Var(\hat{\theta})\geq &\frac{1}{I(\theta)}\\
=& \frac{\sigma^2}{\displaystyle\left(\frac{\partial s(n,\theta)}{\partial \theta}\right)^2}
\end{align}
\end{proof}
\item
设一组观测值$x(n) = A\cos(2\pi f_0 n + \varphi) + w(n),n=0,1,\dots,N-1,0<f_0 <{1\over 2}$,其中$w(n)\sim N(0,\sigma^2)$是高斯白噪声,$A$和$\varphi$已知,通过这组
观测值估计确定量$f_0$,证明:估计值$\hat{f}_0$ 的 Cramer-Rao 下界是:
\begin{equation}
\Var[\hat{f}_0]\geq \frac{\sigma^2}{A^2\displaystyle\sum_{n=0}^{N-1} [2\pi n \sin(2\pi f_0 n + \varphi)]^2}
\end{equation}
\begin{proof}
使用\eqref{eq:CRbound}式的结论,这里$\theta=f_0$,$s(\theta,n)=A\cos(2\pi \theta n+ \varphi)$,即可求出。
\end{proof}
\item
设观测序列为 $x(n),n=0,1,\dots,N-1$,每个样本是独立同分布的,满足如下概率密度函数,
$$
p(x) = \begin{cases} 
\frac{1}{\alpha^2}x\exp(-\frac{x^2}{2\alpha^2}) & x\geq 0 \\
0 & x<0 \\
\end{cases}
$$
设$\alpha>0$,求$\alpha$的MLE。
\begin{solution}
$x(1),\dots,x(N-1)$的
对数似然函数为:
\begin{equation}
\log(p(\bm{x},\alpha))=-2N\log\alpha+\sum_{n=0}^{N-1}\left(\log x(i) - \frac{x(i)^2}{2\alpha^2}\right)
\end{equation}
令$\frac{\partial \log(p(\bm{x},\alpha))}{\partial \alpha}=0$,解得$\alpha$的MLE为
\begin{equation}
\hat{\alpha}=\sqrt{\frac{1}{2N}\displaystyle\sum_{n=0}^{N-1}x(i)^2}
\end{equation}
\end{solution}
\item
设观测样本为 $x(n),n=0,1,\dots,N-1$,每个样本是独立同分布的,且$x(n)$仅取1和0两个值,$\Pr(x(n)=1)=p$未知,通过观测样本求$p$的MLE。
\begin{solution}
$x(1),\dots,x(N-1)$的
对数似然函数为:
\begin{equation}
\log(\Pr(\bm{x},p))=\sum_{n=0}^{N-1} \left(x(i)\log p + (1-x(i))\log(1-p) \right)
\end{equation}
令$\frac{\partial \log(\Pr(\bm{x},p))}{\partial p}=0$,解得$p$的MLE为
\begin{equation}
\hat{p}=\frac{1}{N}\sum_{n=0}^{N-1}x(i)
\end{equation}
\end{solution}
\item
设观测序列为 $x(n) = \theta + w(n),n=0,1,\dots,N-1.$其中$w(n)\sim N(0,\sigma_w^2)$是高斯白噪声,设$\theta$是一个随机参数,服从均匀分布,其概率密度函数为
$$
p(\theta) = \begin{cases}
\frac{1}{\theta_2-\theta_1}  & \theta_1 \leq \theta \leq \theta_2 \\
0 & \textrm{otherwise}\\
\end{cases}
$$
求$\theta $的MAP估计器。
\begin{solution}
MAP Bayes估计器是下面极值问题的解:
\begin{equation}
\hat{\theta}=\rgmax_{\theta} \left(\log(p(x|\theta))+\log(p(\theta))\right)
\end{equation}
去掉无关的常数后有
\begin{equation}
\log(p(x|\theta))+\log(p(\theta))=\log p(\theta)-\sum_{n=0}^{N-1} \frac{(x(n)-\theta)^2}{2\sigma_w^2}
\end{equation}
因为$\theta>\theta_2$或$\theta<\theta_1$时函数为$-\infty$,故只需考虑 $\hat{\theta}\in [\theta_1,\theta_2]$
设$\bar{x}=\frac{1}{N}\displaystyle\sum_{n=0}^{N-1} x(n)$
解得MAP估计器为
\begin{equation}
\hat{\theta}=\begin{cases}
\theta_1,& \bar{x} <\theta_1\\
\bar{x},& \theta_1\leq \bar{x}\leq \theta_2\\
\theta_2,& \bar{x} >\theta_2\\
\end{cases}
\end{equation}
\end{solution}
\end{enumerate}
\end{document}
